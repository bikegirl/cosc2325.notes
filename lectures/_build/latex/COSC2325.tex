%% Generated by Sphinx.
\def\sphinxdocclass{report}
\documentclass[letterpaper,11pt,english]{sphinxmanual}
\ifdefined\pdfpxdimen
   \let\sphinxpxdimen\pdfpxdimen\else\newdimen\sphinxpxdimen
\fi \sphinxpxdimen=.75bp\relax

\usepackage[utf8]{inputenc}
\ifdefined\DeclareUnicodeCharacter
 \ifdefined\DeclareUnicodeCharacterAsOptional\else
  \DeclareUnicodeCharacter{00A0}{\nobreakspace}
\fi\fi
\usepackage{cmap}
\usepackage[T1]{fontenc}
\usepackage{amsmath,amssymb,amstext}
\usepackage{babel}
\usepackage{times}
\usepackage[Bjarne]{fncychap}
\usepackage[dontkeepoldnames]{sphinx}

\usepackage{geometry}

% Include hyperref last.
\usepackage{hyperref}
% Fix anchor placement for figures with captions.
\usepackage{hypcap}% it must be loaded after hyperref.
% Set up styles of URL: it should be placed after hyperref.
\urlstyle{same}
\addto\captionsenglish{\renewcommand{\contentsname}{Contents:}}

\addto\captionsenglish{\renewcommand{\figurename}{Fig.}}
\addto\captionsenglish{\renewcommand{\tablename}{Table}}
\addto\captionsenglish{\renewcommand{\literalblockname}{Listing}}

\addto\extrasenglish{\def\pageautorefname{page}}

\setcounter{tocdepth}{0}



\title{COSC2325 Lecture Notes}
\date{May 29, 2017}
\release{SU17}
\author{Roie R. Black}
\newcommand{\sphinxlogo}{\vbox{}}
\renewcommand{\releasename}{Release}
\makeindex

\begin{document}

\maketitle
\sphinxtableofcontents
\phantomsection\label{\detokenize{index::doc}}



\chapter{Are You Ready to Code?}
\label{\detokenize{01_introduction/are-you-ready-to-code::doc}}
By now, you have had a few programming courses. For a few of you, you have
been writing programs for quite a while. My question is simple: are you ready
to code?

Silly questiton! Of course you are ready to code!

But are you really?


\section{How Do You Code Now?}
\label{\detokenize{01_introduction/are-you-ready-to-code:how-do-you-code-now}}
The answer for many of you is also simple: “In my favorite IDE, which runs on
Windows.”

Sorry! You are not ready to code.


\section{What Is Wrong With My IDE?}
\label{\detokenize{01_introduction/are-you-ready-to-code:what-is-wrong-with-my-ide}}
Nothing really, if you expect to spend a lot of time working in that
environment. The problem is that you may have picked the wrong IDE. When you
get your first real job, you will discover what the right IDE is (for your
employer). You will need to drop what you love now, and learn to love a new
tool. You will also have to learn other new tools and procedures that dictate
how you will write code for that employer.

Oh yeah! You will also learn that your code is going to be scrutinized by your
programming peers. You will face something called the “Peer-Review” where your
code is basically shredded by others, and you get to defend every line you
wrote (Shudder!)


\section{Where Can You Code?}
\label{\detokenize{01_introduction/are-you-ready-to-code:where-can-you-code}}
An even bigger problem lurks out there, ready to bite you! You may not end up
working exclusively on a Windows machine. You might have to work on a Mac, or
on a system running Unix/Linux. Are you ready for that?


\section{A Real Example}
\label{\detokenize{01_introduction/are-you-ready-to-code:a-real-example}}
In my email the other day, I got a summary of the results from the \sphinxhref{https://icpc.baylor.edu/}{ACM-ICPC}
(\sphinxstyleemphasis{International Collegiate Programming Contest}) World Finals, a major
competition to see who has the best programmers in the world. (Hint, the
top-placing US team came in 13th, not a good sign!)

Here is where you would be programming in this competition:
\begin{itemize}
\item {} 
Hardware:
\begin{itemize}
\item {} \begin{description}
\item[{Intel i7 NUC (Model NUC5i7RYHR )}] \leavevmode\begin{itemize}
\item {} 
CPU - 5th Generation Intel Core i7-5557U (3.1+ GHz)

\item {} 
RAM - 16 GB

\item {} 
Disk - 128 GB SSD

\item {} 
Graphics - Intel Iris Graphics 6100

\end{itemize}

\end{description}

\item {} 
External screen: Lenovo ThinkVision L2321x LCD Monitor Model 4014-HB6 (23in wide, 1920x1080 Native Resolution)

\item {} 
External keyboard: Lenovo 41A5100, sub-model KU-0225, USB

\item {} 
External mouse: Lenovo 41U3013, USB

\end{itemize}

\item {} \begin{description}
\item[{Software:}] \leavevmode\begin{itemize}
\item {} \begin{description}
\item[{OS:}] \leavevmode\begin{itemize}
\item {} 
Ubuntu 16.04.1 LTS Linux (64-bit)

\end{itemize}

\end{description}

\item {} \begin{description}
\item[{Desktop:}] \leavevmode\begin{itemize}
\item {} 
GNOME

\end{itemize}

\end{description}

\item {} \begin{description}
\item[{Editors}] \leavevmode\begin{itemize}
\item {} 
vi/vim

\item {} 
gvim

\item {} 
emacs

\item {} 
gedit

\item {} 
geany

\end{itemize}

\end{description}

\item {} \begin{description}
\item[{Languages:}] \leavevmode\begin{itemize}
\item {} \begin{description}
\item[{Java}] \leavevmode\begin{itemize}
\item {} 
OpenJDK version 1.8.0\_91

\item {} 
OpenJDK Runtime Environment (build 1.8.0\_91-8u91-b14-3ubuntu1\textasciitilde{}16.04.1-b14)

\item {} 
OpenJDK 64-Bit Server VM (build 25.91-b14, mixed mode)

\end{itemize}

\end{description}

\item {} \begin{description}
\item[{C}] \leavevmode\begin{itemize}
\item {} 
gcc (Ubuntu 5.4.0-6ubuntu1\textasciitilde{}16.04.2) 5.4.0 20160609

\end{itemize}

\end{description}

\item {} \begin{description}
\item[{C++}] \leavevmode\begin{itemize}
\item {} 
g++ (Ubuntu 5.4.0-6ubuntu1\textasciitilde{}16.04.2) 5.4.0 20160609

\end{itemize}

\end{description}

\item {} \begin{description}
\item[{Python 2}] \leavevmode\begin{itemize}
\item {} 
Python 2.7.10 (implemented using PyPy 5.1.2). A list of the installed modules can be found here

\end{itemize}

\end{description}

\item {} \begin{description}
\item[{Python 3}] \leavevmode\begin{itemize}
\item {} 
CPython 3.5.2.  A list of the installed modules can be found here

\end{itemize}

\end{description}

\end{itemize}

\end{description}

\item {} \begin{description}
\item[{IDEs:}] \leavevmode\begin{itemize}
\item {} 
Java - Eclipse 4.6 (Neon)

\item {} 
C/C++ - CDT 9.0.1 under Eclipse 4.6

\item {} \begin{description}
\item[{Python}] \leavevmode\begin{itemize}
\item {} 
PyDev 5.1.2 under Eclipse 4.6

\item {} 
Pycharm Community Edition 2016.2.3

\end{itemize}

\end{description}

\end{itemize}

\end{description}

\end{itemize}

\end{description}

\end{itemize}

The problems teams faced in this years competition are interesting. Here is
this year’s problem set:
\begin{itemize}
\item {} 
\sphinxcode{/files/icpc2017.pdf}

\end{itemize}

If you were lucky enough to reach this competition, you walked into a room with
nothing but your warm body, and faced the specified working environment. The
competition is timed, and teams work through the problems and submit their code
for testing. Points are awarded for the entries that actually work!

Since this competition is pretty fast paced, I do not see that they considered
program style in their scoring. I certainly would have required a look at that,
perhaps as a last step when awarding points.


\section{Secondary Goal of this Course}
\label{\detokenize{01_introduction/are-you-ready-to-code:secondary-goal-of-this-course}}
This brings me to the secondary goal of this course. I want to turn you into
professional programmers. Well, actually, I want to start you on the path to
becoming one. To do that, we are going to have to un-learn a few bad habits,
and learn some new good-habits. We will also need to learn some new tools, and
we will learn to work in a new environment.

Specifically, here are the tools you will be using in this class (note: no IDEs
allowed!):
\begin{itemize}
\item {} 
Hardware:
\begin{itemize}
\item {} 
Hopefully your personal 64-bit laptop, or a school lab machine

\end{itemize}

\item {} 
Software:
\begin{itemize}
\item {} 
Virtual Machine: VMware or VirtualBox

\item {} 
OS: Ubuntu linux

\item {} \begin{description}
\item[{Languages:}] \leavevmode\begin{itemize}
\item {} 
GNU C/C++

\item {} 
Assembly language (several)

\end{itemize}

\end{description}

\item {} 
Editor: \sphinxhref{http://www.vim.org}{Vim} (or one of its variants)

\item {} 
Source Code Control System: \sphinxhref{https://git-scm.com/}{Git} and \sphinxhref{https://github.com}{Github}

\item {} 
Build tool: GNU \sphinxhref{http://www.gnu.org/software/make/}{Make}

\item {} 
Documentation system: Python \sphinxhref{http://www.sphinx-doc.org/}{Sphinx} or \sphinxhref{https://www.latex-project.org/}{LaTeX}

\end{itemize}

\end{itemize}


\section{DON’T PANIC!}
\label{\detokenize{01_introduction/are-you-ready-to-code:don-t-panic}}
I do not expect you to become wizards in all of this, for most of you, you have
never heard of many of these items. We will introduce each one slowly, and get
you working with them as quickly as we can. I do recommend that you construct
“cheat-sheets” to help you learn how each tool is used.

All of these tools have been in use in this course for many years. A few
students have trouble with some of them, but I am here to help you through
that. I have had quite a few old students contact me years later, thanking me
for pushing them in this direction. Once they got into “real jobs” they
discovered that being a professional programmer is not as easy as they thought
it would be! Having a good start is important, and \sphinxstyleemphasis{NOW} is the time to get
that start!


\section{Meet Guido}
\label{\detokenize{01_introduction/are-you-ready-to-code:meet-guido}}
Although it is not politically correct, I like to introduce a character in my
courses. His name is Guido, thanks to my growing up watching old Mafia TV
shows. With aplogies to every Guido out there, this Guido has to maintain your
code, and he has a very short fuse! He is not someone you want to cross! If you
write code that Guido is going to have a hard time maintaining, he will hunt
you down! You do NOT want that to happen, so think about Guido as you craft
your code.

Guido know his stuff, and he expects you to write nice, clean, well documented
(and maintainable) code that \sphinxstyleemphasis{works}. Nothing else will do!


\section{What is a Working Program Worth?}
\label{\detokenize{01_introduction/are-you-ready-to-code:what-is-a-working-program-worth}}
If you write a program that solves a problem I pose in this class, and it
works, it is worth a “C”.

What? Why is that?

I expect your code to work. That is what I expect from all programmers I have
ever worked with, or who worked for me. You have to do better than that to get
that “A” you thought you should get.

To get the “A”, your code has to be neat (follow accepted style guidelines),
clear (good choices in names you choose), it has to compile and run error free
(even warnings are a bad thing). It has to be documented (not just with
embedded comments). It has to be tested (one sample run is not enough).

Oh! And Guido has to like what you wrote!

You will not get to this point without putting in some effort. I will get
progressively more picky about all of this as the course progresses. For now,
just think about how you currently work, and what habits you might need to
break.


\section{Why?}
\label{\detokenize{01_introduction/are-you-ready-to-code:why}}
There is a point to all of this picky stuff. I want you to enjoy what you do,
and end up working in a job where you are contributing in a real way to the
development of something cool.

You will not get to that place by being sloppy at what you do. That “A” you are
convinced is essential to getting you that job will do no such thing. It may
get you standing in front of a potential employer, but the “A” will get you no
further. It is what you can demonstrate that you \sphinxstyleemphasis{know} and \sphinxstyleemphasis{can do} that will
get you that job. Most employers will run you through a testing process to
prove you can do the job. (I once faced a grueling six hour interview with a
team of folks who wanted to make sure I could run the entire IT department for
a major city. I aced that test because I practice what I preach!)

Programming can and should be enjoyable, and you should be proud of what you
can do! Otherwise, you are headed in the wrong direction.

You will spend an enormous amount of your life working at something. You better
enjoy that! In my over 50 years of working, I have not had one job I would not
gladly go back to and work there again. Maybe I was lucky, but I believe that
having the right attitude, and intentionally developing your skills will get
you a career like that as well.


\chapter{Working on the Command Line}
\label{\detokenize{appendix/working-on-the-command-line::doc}}\label{\detokenize{appendix/working-on-the-command-line:working-on-the-command-line}}
In this class, you will not spend much of your time working with the mouse in one hand. Instead, we will learn how to manage your system and development activities using just the keyboard.

Why?

Modern software is not developed solely on your friendly Windows platform, and
not everything can be done using the mouse. For some of you the mere thought of
letting go of that friendly object strikes you with fear.

Fear not! I actually had the opportunity to put my hands on the first mouse
which came with the first machine that used this critter for anything. The
machine was a Xerox Alto, which I saw at Carnegie Mellon University in the
1970’s.

\noindent{\hspace*{\fill}\sphinxincludegraphics{{Xerox_Alto}.jpg}\hspace*{\fill}}

\begin{sphinxadmonition}{note}{Note:}
The normal screen was paper-white (what else would you expect from Xerox?)
and the cursor was a little guy on skis. People would grab the mouse and
make him ski down the screen. It was entertaining!
\end{sphinxadmonition}

Then Steve and Bill got a look at this machine, and the rest is history.

Before the Alto, all we used to control the computer, or write software, was a
keyboard. We can still do that, but we do so using a special program called the
“shell” or “terminal” on Linux and Mac, and the “command-prompt” window on
Windows.

This is an important skill for a developer. There will be times when the system
you need to work on simply does not have a graphical interface, and your only
way to deal with that machine will be using the command line.

\begin{sphinxadmonition}{note}{Note:}
The term “command line” comes from a line where you type in a “command” to
the operating system. That command directs the system to do something, and
usually results in some program running on the system. What that program is
depends on what you need at the moment.
\end{sphinxadmonition}


\chapter{Creating a Project}
\label{\detokenize{appendix/project-layout:creating-a-project}}\label{\detokenize{appendix/project-layout::doc}}

\chapter{The Development Timeline}
\label{\detokenize{appendix/development-timeline::doc}}\label{\detokenize{appendix/development-timeline:the-development-timeline}}

\chapter{Documenting Your Project}
\label{\detokenize{appendix/documentation:documenting-your-project}}\label{\detokenize{appendix/documentation::doc}}

\chapter{Git Introduction}
\label{\detokenize{appendix/git-introduction:git-introduction}}\label{\detokenize{appendix/git-introduction::doc}}
What?

You just spent all day making changes to a major project that was working just
fine, and now nothing works! You need to go back to yesterday and start over.

Can you do that?

Odds are the answer for many of you is: No! It will take another day to recover
from this disaster!

\sphinxhref{https://en.wikipedia.org/wiki/Linus\_Torvalds}{Linus Torvalds} manages over 1000 programmers who contribute to the
{\color{red}\bfseries{}Linux\_} kernel code, the hert of {\color{red}\bfseries{}Linux\_}. He certainly cannot afford to let any
one of those programmers break the project (which consists of hyndreds of
directories, and thousands of files, BTW!)

WHoe there were many tools available to help with this problem, he was not satisfied with any of them. So, likle any good programmer, he write his own.

The tool he developed is called \sphinxhref{https://git-scm.com/}{Git}, and today \sphinxhref{https://git-scm.com/}{Git} is one of the most popular \sphinxstyleemphasis{source code control system} tools around. BEst of all, it is free!

We will be using Git to manage all of the code you create for this class. With
any luck, you will never write another line of code that is not being managed
by Git, ever again!

\begin{sphinxadmonition}{note}{Note:}
Well, things do change. Maybe there will be a cooler tool that replaces
\sphinxhref{https://git-scm.com/}{Git} some day!
\end{sphinxadmonition}

The {\color{red}\bfseries{}Gnu\_} project has developed a collection of incredible free software
development and system management tools. If you do not know about {\color{red}\bfseries{}{}`Richard
Stallman{}`\_} and his {\color{red}\bfseries{}{}`Free Software Foundation{}`\_}, you should look into it if you
indent to develop software seriously. Although most of these tools are now
associated with the {\color{red}\bfseries{}Linux\_} operating system, Richard initially set out to
produce a completely free version of the Unix operating system. Today, you can
find these tools on just about any platform.

We will be using the \sphinxstyleemphasis{Gnu Compiler collection (GCC\_)} to develop projects for
this class. Specifically, we will use these major tools:
\begin{itemize}
\item {} 
\sphinxstylestrong{gcc} - the Gnu “C” compiler

\item {} 
\sphinxstylestrong{g++} - The Gnu “C++” compiler

\item {} 
\sphinxstylestrong{make} - the Gnu build tool”

\end{itemize}


\chapter{The Vim Programmer’s Editor}
\label{\detokenize{appendix/vim-introduction::doc}}\label{\detokenize{appendix/vim-introduction:the-vim-programmer-s-editor}}
You absolutely must learn how to use a good cross-platform editor, not just the
one you currently use that only works on Windows. In my over 50 years of
programming, the one tool I have been able to use on just about any computer
system I have ever touched is a version on Vim, which is commonly installed on
all Unix/Linux machines. Vim is readily available for Windows ans Mac users as
well.

Picking an editor is a religious affair. One developer might love editor X, and another, editory Y. They will go head to head debating the relative merits of each.

When you leave this class, you pick the editor you want to use. It is you wou will ahve to use this thing, and you should learn it well. My rules for picking one is based on the simple fact that I work on a huge variety of systems, and Vim has always been there for me. YMMV!

I will ask you to use Vim in this class, just to give your the experience in using it. DOn’t panic, it is not that hard to learn. I does have one significant quirk, though. It is called a “modal” editor, because at any given moment in time, Vim is operating in one of two major modfes. More on that later.


\section{Installing Vim}
\label{\detokenize{appendix/vim-introduction:installing-vim}}
Before we can do much with this editor, we need to install it. I will show how
to install it on all three major platforms. After that, we will do a short
introduction int it’s use.


\subsection{Installing on Windows}
\label{\detokenize{appendix/vim-introduction:installing-on-windows}}

\subsection{Installing on Mac}
\label{\detokenize{appendix/vim-introduction:installing-on-mac}}

\subsection{Installing on Ubuntu Linux}
\label{\detokenize{appendix/vim-introduction:installing-on-ubuntu-linux}}
This is pretty simple.


\section{Vim Configuration}
\label{\detokenize{appendix/vim-introduction:vim-configuration}}
Vim uses a control file to let the user customize how it works. This is an
important aspect of a good programmer’s editor. The name and location of the
control file varies with the system.


\subsection{Windows Configuration}
\label{\detokenize{appendix/vim-introduction:windows-configuration}}

\chapter{The Make BUilt Tool}
\label{\detokenize{appendix/make-introduction::doc}}\label{\detokenize{appendix/make-introduction:the-make-built-tool}}

\chapter{The Gnu Compiler Collection}
\label{\detokenize{appendix/the-gnu-compiler-collection::doc}}\label{\detokenize{appendix/the-gnu-compiler-collection:the-gnu-compiler-collection}}
The {\color{red}\bfseries{}Gnu\_} project has developed a collection of incredible free software
development and system management tools. If you do not know about \sphinxhref{https://stallman.org/}{Richard
Stallman} and his \sphinxhref{http://www.fsf.org/}{Free Software Foundation}, you should look into it if you
indent to develop software seriously. Although most of these tools are now
associated with the {\color{red}\bfseries{}Linux\_} operating system, Richard initially set out to
produce a completely free version of the Unix operating system. Today, you can
find these tools on just about any platform.

We will be using the \sphinxstyleemphasis{Gnu Compiler collection (GCC\_)} to develop projects for
this class. Specifically, we will use these major tools:
\begin{itemize}
\item {} 
\sphinxstylestrong{gcc} - the Gnu “C” compiler

\item {} 
\sphinxstylestrong{g++} - The Gnu “C++” compiler

\item {} 
\sphinxstylestrong{make} - the Gnu build tool”

\end{itemize}


\chapter{Indices and tables}
\label{\detokenize{index:indices-and-tables}}\begin{itemize}
\item {} 
\DUrole{xref,std,std-ref}{genindex}

\item {} 
\DUrole{xref,std,std-ref}{modindex}

\item {} 
\DUrole{xref,std,std-ref}{search}

\end{itemize}



\renewcommand{\indexname}{Index}
\printindex
\end{document}
